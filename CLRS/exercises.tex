\documentclass{article}

\usepackage{algorithm}
\usepackage{algpseudocode}

\title{Solutions to the exercices in CLRS}
\author{Andrianambinina Marius Rabenarivo}

\begin{document}

\maketitle

1.1-1
Give a real-world example that requires sorting or a real-world example that re-
quires computing a convex hull.

sorting : order lines of a table relative to a specific column

1.1-2
Other than speed, what other measures of efficiency might one use in a real-world setting?

memory usage

1.1-3
Select a data structure that you have seen previously, and discuss its strengths and limitations.

array

stength : easy random access (we can get an element by it's index)
limitation : fixed length

1.1-4
How are the shortest-path and traveling-salesman problems given above similar?
How are they different?

shortest-path and traveling-salesman problems given above are similar
according to the fact that they both involve reducing the length of
the path to be traversed.

They are different as the shortest path has an unique destination
and traveling-salesman has multiple destination to be visited.

1.1-5
Come up with a real-world problem in which only the best solution will do. Then
come up with one in which a solution that is “approximately” the best is good
enough.

surgery

video streaming


1.2-1

finding route followed by buses in the urban area of Antananarivo

extrapolate route used by buses relative to coordinates sent by users


4.1-1
What does FIND-MAXIMUM-SUBARRAY return when all elements of A are negative?

The biggest number (nearest to 0).

4.1-2
Write pseudocode for the brute-force method of solving the maximum-subarray
problem. Your procedure should run in $\theta(n^2)$ time.

\begin{algorithm}
  \begin{algorithmic}
    \Procedure{MAX\_SUBARRAY\_QUAD}{A}
      \For{$i \gets 1, A.size$}
        \For{$j \gets A.size, i$}
          \For{$k \gets i, j$}
            \State $sum \gets sum + A[k]$
          \EndFor
          \If{$sum > max\_sum$}
            \State $max\_sum \gets sum$
          \EndIf
        \EndFor
      \EndFor
    \EndProcedure
  \end{algorithmic}
\end{algorithm}

\end{document}


4.1-4
Suppose we change the definition of the maximum-subarray problem to allow the
result to be an empty subarray, where the sum of the values of an empty subarray is 0. How would you change any of the algorithms that do not allow empty
subarrays to permit an empty subarray to be the result?

I woud a test if the result is negative to output the empty subarray to be the result.
